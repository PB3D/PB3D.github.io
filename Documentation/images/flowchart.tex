% See https://tex.stackexchange.com/a/51761/78675

\documentclass[x11names,crop,tikz]{standalone}% 'crop' is the default for v1.0, before it was 'preview'
\usepackage{tikz}
\usetikzlibrary{shapes,arrows,chains}
\RequirePackage[no-math]{fontspec}
    \defaultfontfeatures{Ligatures=TeX}
    \setsansfont{Lato}
    \renewcommand{\familydefault}{\sfdefault}
\begin{document}

% Adapted from http://www.texample.net/tikz/examples/flexible-flow-chart/
% =================================================
% Set up a few colours
\colorlet{lcfree}{Green3}
%\colorlet{lcnorm}{Blue3}
\definecolor{lcnorm}{HTML}{3D578C} 
\colorlet{lccong}{Red3}
% -------------------------------------------------
% Set up a new layer for the debugging marks, and make sure it is on
% top
\pgfdeclarelayer{marx}
\pgfsetlayers{main,marx}
% A macro for marking coordinates (specific to the coordinate naming
% scheme used here). Swap the following 2 definitions to deactivate
% marks.
\providecommand{\cmark}[2][]{%
  \begin{pgfonlayer}{marx}
    \node [nmark] at (c#2#1) {#2};
  \end{pgfonlayer}{marx}
  } 
\providecommand{\cmark}[2][]{\relax} 
\def \boxwidth {45mm}
\def \boxheight {8mm}
% -------------------------------------------------
% Start the picture
\begin{tikzpicture}[%
    >=triangle 60,              % Nice arrows; your taste may be different
    start chain=going below,    % General flow is top-to-bottom
    node distance=\boxheight and \boxwidth, % Global setup of box spacing
    every join/.style={norm},   % Default linetype for connecting boxes
    ]
% ------------------------------------------------- 
% A few box styles 
% <on chain> *and* <on grid> reduce the need for manual relative
% positioning of nodes
\tikzset{
  base/.style={draw, on chain, on grid, align=center, minimum height=4ex},
  proc/.style={base, rectangle, text width=8em},
  test/.style={base, diamond, aspect=1.5, text width=5em},
  term/.style={proc, rounded corners},
  % coord node style is used for placing corners of connecting lines
  coord/.style={coordinate, on chain, on grid, node distance=\boxheight and 0.3*\boxwidth},
  % nmark node style is used for coordinate debugging marks
  nmark/.style={draw, cyan, circle, font={\sffamily\bfseries}},
  % -------------------------------------------------
  % Connector line styles for different parts of the diagram
  norm/.style={->, draw, lcnorm},
  free/.style={->, draw, lcfree},
  cong/.style={->, draw, lccong},
  it/.style={font={\small\itshape}}
}
% -------------------------------------------------
% Start by placing the nodes
% Use join to connect a node to the previous one
% No join for exits from test nodes - connections have more complex
% requirements
\node [proc, densely dotted, it] (p0) {PB3D initialize};
\node [proc, densely dotted, it] (p5) [below=2cm of p0] {PB3D finalize};
\node [test] (t0) [right=of p5] {Richardson converged?};
\node [proc] (p4) [below=2cm of t0] {solution driver};
\node [proc] (p2) [right=of t0] {equilibrium driver};
\node [proc, join] (p3) [right=of p4] {perturbation driver};
\node [proc] (p1) [right=2*\boxwidth of p0] {input driver};
% -------------------------------------------------
% Now we place the coordinate nodes for the connectors with angles, or
% with annotations. We also mark them for debugging.
% \node [coord, right=of p0] (c1)  {}; \cmark{1}   
% \node [coord, left=of p1] (c2)  {}; \cmark{2}   
% % \node [coord, right=of t3] (c3)  {}; \cmark{3}   
% % \node [coord, right=of t6] (c6)  {}; \cmark{6}   
% % \node [coord, right=of t7] (c7)  {}; \cmark{7}   
% % \node [coord, left=of t4]  (c4)  {}; \cmark{4}   
% % \node [coord, right=of t4] (c4r) {}; \cmark[r]{4}
% % \node [coord, left=of t7]  (c5)  {}; \cmark{5}   
% -------------------------------------------------
% A couple of boxes have annotations
% % \node [above=0mm of p4, it] {(Queue was empty)};
% % \node [above=0mm of p8, it] {(Queue was not empty)};
% -------------------------------------------------
% All the other connections come out of tests and need annotating
% First, the straight north-south connections. In each case, we first
% draw a path with a (consistently positioned) annotation node, then
% we draw the arrow itself.
% % \path (t1.south) to node [near start, xshift=1em] {$y$} (p2);
% %   \draw [*->,lcnorm] (t1.south) -- (p2);
% % \path (t2.south) to node [near start, xshift=1em] {$y$} (t3); 
% %   \draw [*->,lcnorm] (t2.south) -- (t3);
% % \path (t3.south) to node [near start, xshift=1em] {$y$} (t4); 
% %   \draw [*->,lcnorm] (t3.south) -- (t4);
% % \path (t5.south) to node [near start, xshift=1em] {$y$} (t6); 
% %   \draw [*->,lcfree] (t5.south) -- (t6);
% % \path (t6.south) to node [near start, xshift=1em] {$y$} (t7); 
% %   \draw [*->,lcfree] (t6.south) -- (t7); 
% ------------------------------------------------- 
% Now the straight east-west connections. To provide consistent
% positioning of the test exit annotations, we have positioned
% coordinates for the vertical part of the connectors. The annotation
% text is positioned on a path to the coordinate, and then the whole
% connector is drawn to its destination box.
% \path (p0.east) to node [near start, yshift=1em] {} (c2); 
  \draw [->,lcnorm] (p0.east) -- (p1.west);
% \path (p1.south) to node [near start, yshift=1em] {} (p2.north); 
  \draw [->,lcnorm] (p1.south) -- (p2.north);
\path (t0.east) to node [yshift=1em] {\textit{refine}} (p2.west); 
  \draw [*->,lcnorm] (t0.east) -- (p2.west);
\path (t0.west) to node [near start, yshift=1em] {\textit{R.A.}} (p5.east); 
  \draw [o->,lcnorm] (t0.west) -- (p5.east);
\path (p2.south) to node [xshift=2.1em] {\textit{eq. jobs}} (p3.north); 
  \draw [<->,lccong] (p2.south) -- (p3.north);
% \path (p1.south) to node [near start, yshift=1em] {} (p2.north); 
  \draw [->,lcnorm] (p3.west) -- (p4.east);
% \path (p1.south) to node [near start, yshift=1em] {} (p2.north); 
  \draw [->,lcnorm] (p4.north) -- (t0.south);

% \path (p0.east) to node [near start, yshift=1em] {$n$} (c2); 
%   \draw [o->,lcnorm] (p0.east) -- (p1);
% % \path (t4.east) to node [yshift=-1em] {$k \leq 0$} (c4r); 
% %   \draw [o->,lcnorm] (t4.east) -- (p9);
% -------------------------------------------------
% Finally, the twisty connectors. Again, we place the annotation
% first, then draw the connector
% % \path (t1.east) to node [near start, yshift=1em] {$n$} (c1); 
% %   \draw [o->,lcfree] (t1.east) -- (c1) |- (p4);
% % \path (t2.east) -| node [very near start, yshift=1em] {$n$} (c1); 
% %   \draw [o->,lcfree] (t2.east) -| (c1);
% % \path (t4.west) to node [yshift=-1em] {$k>0$} (c4); 
% %   \draw [*->,lcnorm] (t4.west) -- (c4) |- (p3);
% % \path (t5.east) -| node [very near start, yshift=1em] {$n$} (c6); 
% %   \draw [o->,lcfree] (t5.east) -| (c6); 
% % \path (t6.east) to node [near start, yshift=1em] {$n$} (c6); 
% %   \draw [o->,lcfree] (t6.east) -| (c7); 
% % \path (t7.east) to node [yshift=-1em] {$k \leq 0$} (c7); 
% %   \draw [o->,lcfree] (t7.east) -- (c7)  |- (p9);
% % \path (t7.west) to node [yshift=-1em] {$k>0$} (c5); 
% %   \draw [*->,lcfree] (t7.west) -- (c5) |- (p5);
% -------------------------------------------------
% A last flourish which breaks all the rules
% % \draw [->,MediumPurple4, dotted, thick, shorten >=1mm]
% %   (p9.south) -- ++(5mm,-3mm)  -- ++(27mm,0) 
% %   |- node [black, near end, yshift=0.75em, it]
% %     {(When message + resources available)} (p0);
% -------------------------------------------------
\end{tikzpicture}
% =================================================




\end{document}

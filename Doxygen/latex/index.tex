{\itshape The main presentation of the P\+B3D project information can be found at the \href{https://pb3d.github.io}{\texttt{ P\+B3D home}}.}

 
\textit{\textbf{Be sure to also consult the \href{https://pb3d.github.io/Doxygen/html/index.html}{\tt html version} of this document.}}


This page contains the technical documentation of the {\bfseries{P\+B3D}} project. A pdf version can be downloaded \href{https://pb3d.github.io/PB3D_manual.pdf}{\texttt{ here}}.

It contains a general overview of the code in \mbox{\hyperlink{page_overview}{General code structure}}, which is much more detailed than what can be found in \cite{Weyens2017PB3D}.

To get started, for installation instructions, please see \mbox{\hyperlink{page_installation}{Installation}} instructions.

Also, in \mbox{\hyperlink{page_tutorial}{Tutorial}}, you can find a hands-\/on tutorial on running the code.

Furthermore, in \mbox{\hyperlink{page_inputs}{Input variables}}, the multitude of input variables is discussed.

Additionally, in \mbox{\hyperlink{page_outputs}{Code outputs}}, the same is done for the output that can be produced, including the output by P\+O\+ST.

For a list of frequently asked questions and frequently encountered problems, consult the \mbox{\hyperlink{page_faq}{Frequently Asked Questions}}.

  
Finally, the various modules of which the program consists can be explored in detail in the appendices pages, as well as the procedures and the variables.


\begin{center}{\itshape —this documentation was created with \href{http://www.doxygen.org/index.html}{\texttt{ Doxygen 1.\+8.\+17}}—}\end{center}  
\hypertarget{page_installation_installation_introduction}{}\section{Introduction}\label{page_installation_installation_introduction}
P\+B3D is written in Fortran, and makes use of multiple numerical libraries\+:
\begin{DoxyItemize}
\item \href{http://www.netlib.org/lapack/}{\tt blas / lapack}
\begin{DoxyItemize}
\item for basic linear algebra
\end{DoxyItemize}
\item \href{http://www.netlib.org/scalapack/}{\tt pblas / blacs / scalapack}
\begin{DoxyItemize}
\item for parallelized basic linear algebra
\end{DoxyItemize}
\item \href{https://www.hdfgroup.org/HDF5/}{\tt H\+D\+F5}
\begin{DoxyItemize}
\item for storage files
\item works in parallel
\end{DoxyItemize}
\item \href{https://www.unidata.ucar.edu/software/netcdf/}{\tt Net\+C\+DF}
\begin{DoxyItemize}
\item to read input of V\+M\+EC
\item sequential
\end{DoxyItemize}
\item \href{https://www.mcs.anl.gov/petsc/}{\tt P\+E\+T\+Sc} / \href{http://slepc.upv.es/}{\tt S\+L\+E\+Pc}
\begin{DoxyItemize}
\item for linear algebra of large, sparse matrices
\item can reach $\sim \mathcal{O} (n)$ complexity
\item Minimal installation instructions\+:
\begin{DoxyEnumerate}
\item Configure P\+E\+T\+Sc using {\ttfamily ./configure P\+E\+T\+S\+C\+\_\+\+A\+R\+CH=complex C\+O\+P\+T\+F\+L\+A\+GS=-\/\+O3 C\+X\+X\+O\+P\+T\+F\+L\+A\+GS=-\/\+O3 F\+O\+P\+T\+F\+L\+A\+GS=-\/\+O3 \textbackslash{}}~\newline
 {\ttfamily -\/-\/with-\/scalar-\/type=complex \textbackslash{}}~\newline
 {\ttfamily -\/-\/download-\/metis \textbackslash{}}~\newline
 {\ttfamily -\/-\/download-\/mumps \textbackslash{}}~\newline
 {\ttfamily -\/-\/download-\/parmetis \textbackslash{}}~\newline
 {\ttfamily -\/-\/with-\/scalapack-\/lib=\char`\"{}-\/\+L\mbox{[}\+S\+C\+A\+L\+A\+P\+A\+C\+K\+D\+I\+R\mbox{]}/lib -\/lscalapack\char`\"{} \textbackslash{}}~\newline
 {\ttfamily -\/-\/with-\/valgrind-\/dir=/usr \textbackslash{}}~\newline
 {\ttfamily -\/-\/with-\/debugging=no \textbackslash{}}~\newline
 {\ttfamily -\/-\/with-\/fortran-\/kernels=1}~\newline
 (If you want debug, remove {\ttfamily -\/-\/with-\/debugging=no}, {\ttfamily -\/-\/with-\/fortran-\/kernels=1}, C\+O\+P\+T\+S\+F\+L\+A\+GS, C\+X\+X\+O\+P\+T\+F\+L\+A\+GS and F\+O\+P\+T\+F\+L\+A\+GS, and change P\+E\+T\+S\+C\+\_\+\+A\+R\+CH to {\ttfamily debug-\/complex}.)~\newline
 (If you use a different H\+D\+F5, add {\ttfamily -\/-\/with-\/hdf5-\/dir=\mbox{[}H\+D\+F5\+D\+IR\mbox{]}}.)
\item follow instructions to do makes and tests.
\item Set global variables {\ttfamily export S\+L\+E\+P\+C\+\_\+\+D\+IR=\mbox{[}S\+L\+E\+PC D\+IR\mbox{]}}, {\ttfamily export P\+E\+T\+S\+C\+\_\+\+D\+IR=\mbox{[}P\+E\+T\+S\+C\+\_\+\+D\+IR\mbox{]}} and {\ttfamily export P\+E\+T\+S\+C\+\_\+\+A\+R\+CH=\mbox{[}debug-\/\mbox{]}complex} (depending on whether it is debug or not).
\item Configure S\+L\+E\+Pc using {\ttfamily ./configure}
\item follow instructions to do makes and tests
\end{DoxyEnumerate}
\end{DoxyItemize}
\item \href{http://portal.nersc.gov/project/sparse/strumpack/}{\tt Strum\+Pack}
\begin{DoxyItemize}
\item for linear algebra of structured matrices \cite{Ambikasaran2013}
\item the vacuum potential energy perturbation calculations use such matrices as they are generated with a $1/r$ kernel.
\item can reach $\sim \mathcal{O} (n \log(n))$ complexity
\item we use the dense version of the package, which for Fortran is 1.\+1.\+1.
\item Minimal installation instructions\+:
\begin{DoxyEnumerate}
\item \href{http://portal.nersc.gov/project/sparse/strumpack/STRUMPACK-Dense-1.1.1.tar.gz}{\tt Download}, untar and go to root folder.
\item Go to examples, copy {\ttfamily Makefile.\+gnu} to {\ttfamily Makefile.\+inc} and adapt\+:
\begin{DoxyEnumerate}
\item for example, it might be necessary to add {\ttfamily -\/lblacs} before {\ttfamily -\/lscalapack}
\item if you use intel, have a look at the \href{https://software.intel.com/en-us/articles/intel-mkl-link-line-advisor}{\tt link line advisor}, and update the {\ttfamily L\+IB}
\item possibly, you might have to set the {\ttfamily CC} and {\ttfamily FC} compiler commands to, for example, {\ttfamily mpicc} respectively {\ttfamily mpiifort}
\end{DoxyEnumerate}
\item Create directories {\ttfamily lib} and {\ttfamily inc}.
\item Run {\ttfamily make f90\+\_\+example}
\item Make a library of {\ttfamily Strumpack\+Dense\+Package.\+o} using {\ttfamily ar -\/rcs libstrumpack.\+a $\ast$.o} in src.
\item In inc, create a symlink {\ttfamily libstrumpack.\+a} to {\ttfamily libstrumpack.\+a} in src.
\item In lib, create a symlink {\ttfamily strumpackdensepackage.\+mod} to {\ttfamily strumpackdensepackage.\+mod} in examples.
\end{DoxyEnumerate}
\end{DoxyItemize}
\item \href{http://vmecwiki.pppl.wikispaces.net/STELLOPT+Compilation}{\tt libstell}
\begin{DoxyItemize}
\item part of Stellopt suite, which contains V\+M\+EC
\item provides routines to read V\+M\+EC output data
\end{DoxyItemize}
\item \href{https://w3.pppl.gov/ntcc/PSPLINE/}{\tt pspline}
\begin{DoxyItemize}
\item Princeton Spline and Hermite Cubic Interpolation Routines
\item Minimal installation instructions\+:
\begin{DoxyEnumerate}
\item Export {\ttfamily F\+F\+L\+A\+GS} en {\ttfamily C\+F\+L\+A\+GS} if you want to optimize\+:
\begin{DoxyItemize}
\item copy {\ttfamily share/\+Make.\+overwrite.\+sample} to {\ttfamily share/\+Make.\+overwrite}
\item Run the gmake below and note the flags, then edit these and put them in {\ttfamily share/\+Make.\+overwrite}
\item if G\+CC\+:
\begin{DoxyItemize}
\item {\ttfamily F\+F\+L\+A\+GS = -\/c -\/\+O3 -\/m64 -\/fno-\/range-\/check -\/fdollar-\/ok -\/cpp ; export F\+F\+L\+A\+GS}
\item {\ttfamily C\+F\+L\+A\+GS = -\/c -\/\+O3 -\/m64 ; export C\+F\+L\+A\+GS}
\item {\ttfamily C\+X\+X\+F\+L\+A\+GS = -\/c -\/\+O3 -\/m64 ; export C\+X\+X\+F\+L\+A\+GS}
\end{DoxyItemize}
\item if Intel\+:
\begin{DoxyItemize}
\item {\ttfamily F\+F\+L\+A\+GS = -\/c -\/\+O3 -\/nowarn -\/ftz -\/auto-\/scalar -\/traceback -\/align dcommons}
\item {\ttfamily M\+P\+I\+\_\+\+F\+F\+L\+A\+GS = -\/c -\/\+O3 -\/nowarn -\/ftz -\/traceback -\/align dcommons -\/auto-\/scalar} (possibly add FC=mpiifort, {\ttfamily C\+XX=mpicpc}, {\ttfamily CC=mpiicc} if on a cluster)
\end{DoxyItemize}
\end{DoxyItemize}
\item Compile with {\ttfamily gmake N\+E\+T\+C\+D\+F\+\_\+\+D\+IR= F\+O\+R\+T\+R\+A\+N\+\_\+\+V\+A\+R\+I\+A\+NT=\mbox{[}V\+A\+R\+I\+A\+NT\mbox{]}} with {\ttfamily \mbox{[}V\+A\+R\+I\+A\+NT\mbox{]}} either {\ttfamily G\+CC} or {\ttfamily Intel} (add {\ttfamily D\+E\+B\+UG=1} and remove the F\+L\+A\+GS if debug wanted). Possibly also add {\ttfamily O\+BJ=\$\mbox{[}C\+O\+M\+P\+\_\+\+D\+IR\mbox{]}} if on a cluster, with {\ttfamily \mbox{[}C\+O\+M\+P\+\_\+\+D\+IR\mbox{]}} the directory where to put the resulting library.
\item Run tests from {\ttfamily R\+E\+A\+D\+M\+E\+\_\+\+Pspline}.
\end{DoxyEnumerate}
\end{DoxyItemize}
\end{DoxyItemize}

They should probably be installed in this order. On linux distributions such as Ubuntu, they may be available as packages.

Furthermore, P\+B3D comes bundled with some other, smaller libraries\+:
\begin{DoxyItemize}
\item \href{http://www.netlib.org/fftpack/}{\tt fftpack}
\begin{DoxyItemize}
\item to calculate the fast Fourier transform
\end{DoxyItemize}
\item \href{http://foul.sourceforge.net/}{\tt foul}
\begin{DoxyItemize}
\item the Fortran Output Library
\end{DoxyItemize}
\end{DoxyItemize}

These do not have to be installed separately.\hypertarget{page_installation_installation_compilation}{}\section{Compilation}\label{page_installation_installation_compilation}
When all dependencies are satisfied, the program is then compiled in the standard way\+:
\begin{DoxyItemize}
\item Including the headers of all the libraries in the compilation of the object files\+:
\begin{DoxyItemize}
\item This is done using {\ttfamily -\/I\mbox{[}path\+\_\+to\+\_\+library\mbox{]}}.
\item Make sure you add the {\ttfamily -\/o} option to create only object files.
\end{DoxyItemize}
\item Linking with the actual libraries
\begin{DoxyItemize}
\item This is done using {\ttfamily -\/L\mbox{[}path\+\_\+to\+\_\+library\mbox{]} -\/l\mbox{[}library\+\_\+name\mbox{]}}.
\end{DoxyItemize}
\end{DoxyItemize}\hypertarget{page_installation_installation_makefile}{}\section{Makefile Example}\label{page_installation_installation_makefile}

\begin{DoxyCodeInclude}
1 ##############################################################################
2 #
3 #   Example makefile for the program PB3D (Peeling Ballooning in 3D)
4 #   \(\backslash\)author Author: Toon Weyens
5 #
6 #   Don't forget to set the directories:
7 #       - LIBSTELL\_DIR
8 #       - HDF5\_DIR
9 #       - NETCDFF\_DIR (note: Fortran library)
10 #       - PETSC\_DIR
11 #       - SLEPC\_DIR
12 #       - STRUMPACK\_DIR
13 ##############################################################################
14 
15 ##############################################################################
16 #   Include
17 ##############################################################################
18 ## [PETSc and SLEPc trick]
19 include  $(PETSC\_DIR)/lib/petsc/conf/variables
20 include  $(SLEPC\_DIR)/lib/slepc/conf/slepc\_variables
21 ## [PETSc and SLEPc trick]
22 
23 ## [PETSc and SLEPc trick inc]
24 INCLUDE = $(PETSC\_FC\_INCLUDES) $(SLEPC\_INCLUDE)
25 ## [PETSc and SLEPc trick inc]
26 ## [Libstell special]
27 INCLUDE += -I$(LIBSTELL\_DIR)/libstell\_dir
28 ## [Libstell special]
29 INCLUDE += -I$(STRUMPACK\_DIR)/include
30 ## [PB3D include]
31 INCLUDE += -I$(PB3D\_DIR)/include
32 ## [PB3D include]
33 INCLUDE += -I/usr/include/hdf5/openmpi
34 
35 ##############################################################################
36 #   Link
37 ##############################################################################
38 ## [PB3D libraries]
39 LIB\_INTERNAL = libdfftpack.a libfoul.a libbspline.a
40 ## [PB3D libraries]
41 
42 LINK := $(LIB\_INTERNAL)
43 
44 ## [PETSc and SLEPc trick lib]
45 LINK += $(PETSC\_LIB)
46 LINK += $(SLEPC\_LIB)
47 ## [PETSc and SLEPc trick lib]
48 LINK += $(LIBSTELL\_DIR)/libstell.a
49 LINK += -L$(STRUMPACK\_DIR)/lib -lstrumpack
50 LINK += -L$(HDF5\_DIR) -lhdf5\_fortran -lhdf5
51 LINK += -L$(NETCDFF\_DIR)/lib -lnetcdff
52 LINK += -Wl,-R$(NETCDFF\_DIR)/lib
53 LINK += -lscalapack -lblacs -lblas -lm
54 LINK += -lstdc++ -lmpi\_cxx
55 
56 
57 ##############################################################################
58 #   Compiler
59 ##############################################################################
60 COMPILER=mpifort
61 
62 
63 ##############################################################################
64 #   Linker
65 ##############################################################################
66 LINKER=mpifort
67 
68 
69 ##############################################################################
70 #   Compiler flags
71 #   options (used with -D[name]):
72 #       ldebug: debug
73 #       lIB: infiniband
74 #       lwith\_gnu: use GNU compiler [default]
75 #       lwith\_intel: use INTEL compiler, (checked for version 12.0.2)
76 #   note: INTEL warning 6536 is suppressed, which informs about extra "USE".
77 #   note: INTEL warning 6843 is suppressed, which informs about empty
78 #       intent(out) variables
79 ##############################################################################
80 COMP\_FLAGS = -finit-real=snan -g -Og -Wall -Wextra -pedantic \(\backslash\)
81     -fimplicit-none -fbacktrace -fno-omit-frame-pointer \(\backslash\)
82     -fcheck=all -cpp -Dldebug# debug, profiling with gprof2dot, GCC
83 #COMP\_FLAGS = -O3 -fbacktrace -g -fimplicit-none -fno-omit-frame-pointer \(\backslash\)
84     #-cpp# optimized, GCC
85 
86 #COMP\_FLAGS = -O0 -DlIB -Dldebug -g -heap-arrays 100 -recursive \(\backslash\)
87     #-ftrapuv -check bounds -check uninit -traceback -implicitnone \(\backslash\)
88     #-fno-omit-frame-pointer -cpp -Dlwith\_intel -diag-disable 6536 \(\backslash\)
89     #-diag-disable 6843# debug, profiling with gprof2dot, INTEL
90 #COMP\_FLAGS = -O3 -DlIB -traceback -g -heap-arrays 100 -recursive \(\backslash\)
91     #-implicitnone -fno-omit-frame-pointer -cpp -Dlwith\_intel \(\backslash\)
92     #-diag-disable 6536 -diag-disable 6843# optimized, INTEL
93 
94 COMP\_FLAGS\_EX= -O2 -w
95 
96 COMP\_FLAGS\_F= -O2 -funroll-loops -fexpensive-optimizations
97 
98 
99 ##############################################################################
100 #   Link flags
101 ##############################################################################
102 LINK\_FLAGS = -fPIC -finit-real=snan# debug
103 #LINK\_FLAGS = -fPIC# optimized
104 
105 
106 ##############################################################################
107 #   Prepare
108 ##############################################################################
109 # Add "Modules" and "Libraries" to the search path for the prerequisites
110 VPATH = Modules:Libraries
111 
112 # Contains list of source files (.o) and dependencies
113 DEPLIST = PB3D.dep
114 OBJLIST = ObjectList# defines "ObjectFiles"
115 
116 # Includes source files and dependency list
117 include $(DEPLIST)# Dependencies of all the objects
118 include $(OBJLIST)# Names of all the objects
119 
120 
121 ##############################################################################
122 #   Rules
123 ##############################################################################
124 all:    PB3D POST
125 
126 PB3D:   $(ObjectFiles) $(LIB\_INTERNAL) PB3D.o
127     $(LINKER) -o $@ $(ObjectFiles) PB3D.o $(LINK) $(LINK\_FLAGS)
128 
129 POST:   $(ObjectFiles) $(LIB\_INTERNAL) POST.o
130     $(LINKER) -o $@ $(ObjectFiles) POST.o $(LINK) $(LINK\_FLAGS)
131 
132 libdfftpack.a:  dfft.o
133     ar -rcs libdfftpack.a dfft.o
134 
135 libfoul.a:  foul.o
136     ar -rcs libfoul.a foul.o
137 
138 libbspline.a:   bspline\_sub\_module.o
139     ar -rcs libbspline.a bspline\_sub\_module.o
140 
141 %.o: %.f90
142     $(COMPILER) $(INCLUDE) $(COMP\_FLAGS) -c $<
143 
144 %.o: %.f
145     $(COMPILER) $(COMP\_FLAGS\_F) -c $<
146 
147 dfft.o: dfft.f
148     $(COMPILER) $(COMP\_FLAGS\_EX) -c $<
149 
150 foul.o: foul.f90
151     $(COMPILER) $(COMP\_FLAGS\_EX) -c $<
152 
153 bspline\_sub\_module.o: bspline\_sub\_module.f90
154     $(COMPILER) $(COMP\_FLAGS\_EX) -c $<
155 
156 clean:
157     @rm -f *.o *.a *.mod *~ fort.* 
158 
159 clean\_all:
160     @rm -f *.o *.mod *~ fort.* PB3D POST
\end{DoxyCodeInclude}


\begin{DoxyNote}{Note}

\begin{DoxyEnumerate}
\item P\+E\+T\+Sc and S\+L\+E\+Pc don\textquotesingle{}t like to be included in another makefile. The trick is to include two files\+: 
\begin{DoxyCodeInclude}
19 include  $(PETSC\_DIR)/lib/petsc/conf/variables
20 include  $(SLEPC\_DIR)/lib/slepc/conf/slepc\_variables
\end{DoxyCodeInclude}
 which will load the variables {\ttfamily P\+E\+T\+S\+C\+\_\+\+F\+C\+\_\+\+I\+N\+C\+L\+U\+D\+ES} and {\ttfamily S\+L\+E\+P\+C\+\_\+\+I\+N\+C\+L\+U\+DE}, used in 
\begin{DoxyCodeInclude}
24 INCLUDE = $(PETSC\_FC\_INCLUDES) $(SLEPC\_INCLUDE)
\end{DoxyCodeInclude}
 as well as the variables {\ttfamily P\+E\+T\+S\+C\+\_\+\+L\+IB} and {\ttfamily S\+L\+E\+P\+C\+\_\+\+L\+IB}, used in 
\begin{DoxyCodeInclude}
45 LINK += $(PETSC\_LIB)
46 LINK += $(SLEPC\_LIB)
\end{DoxyCodeInclude}

\item There are versions of libstell that do not use the standard convention. In this case you have to look for the {\ttfamily $\ast$.mod} files. In the example makefile this is done with 
\begin{DoxyCodeInclude}
27 INCLUDE += -I$(LIBSTELL\_DIR)/libstell\_dir
\end{DoxyCodeInclude}
 instead of the standard {\ttfamily inc} directory.
\item In 
\begin{DoxyCodeInclude}
31 INCLUDE += -I$(PB3D\_DIR)/include
\end{DoxyCodeInclude}
 there are includefiles that contain macros and wrappers specifically for P\+B3D.
\item In 
\begin{DoxyCodeInclude}
39 LIB\_INTERNAL = libdfftpack.a libfoul.a libbspline.a
\end{DoxyCodeInclude}
 linking is done with external libraries that are bundled with P\+B3D. 
\end{DoxyEnumerate}
\end{DoxyNote}
